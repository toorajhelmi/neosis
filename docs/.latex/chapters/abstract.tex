% Abstract for LaTeX document
% This will be included in the main.tex abstract environment

This paper presents Neosis, a computational framework for simulating self-modifying computational organisms that evolve through energy-constrained computation. In Neosis, organisms called Neos exist within environments (NeoVerse) and attempt to predict future sensory signals. The accuracy of predictions determines energy rewards, while all computation and structural modifications consume energy, creating a direct link between intelligence and survival. Neos can modify their structure through mutations (bit addition/removal, edge addition/removal, and truth table modifications) via an embedded mutator called Lio. A meta-level mutator, Evo, can further modify Lio itself. We present the framework's architecture, core axioms, and demonstrate its behavior through simulation experiments showing how different mutation strategies affect organism survival and accuracy over time.

